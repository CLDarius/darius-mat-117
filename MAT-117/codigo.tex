\documentclass[¨12pt,letter paper]{article}
\usepackage[utf8]{inputenc}
\usepackage{amsmath}
\usepackage{amsfonts}
\usepackage{amssymb}
\usepackage{enumitem}
\usepackage{graphicx}
\usepackage{amsthm}
\usepackage{enumitem}
\usepackage{multicol}
\usepackage{parskip}
\title{DESAFIO, MAT-117}
\author{Darío Choque L.}
\date{12 de April 2024}
\begin{document}
   \begin{figure}
        \centering
        \includegraphics[width=0.2\linewidth]{logoumsa.png}
        \caption{UMSA}
        \label{fig:enter-label}
    \end{figure}
 \maketitle
         
     \section{DESAFIO }
         \subsection{Medida}  
               \begin{itemize}
                       \item [$-$] Por $espacio$ $medible$ entendemos un par ordenado $ (\Omega ,B)$que consta de un conjunto $\Omega$ y un $\sigma$ $-$álgebra $B$ de subconjuntos de $\Omega$. Un conjunto $A$ de $\Omega$ se llama $medible$ si $A \in B$.
                       \item [$-$] Una $medida$ $\mu$ en un espacio medible $(\Omega ,B)$ es una función $\mu$:$B \longrightarrow [0,\infty]$ que satisface:
                       $$\mu(\varnothing)=0$$
                       $$\mu(\bigcup_{i}^\infty E_i)=\sum_i^{\infty}{ \mu (E_i)}$$
                       para cualquier suseción ${E_i}$ de conjuntos medibles disjuntos, es decir $E_i \cap E_j=\varnothing,E_i \in B,i\neq j$.
                       \item [$-$] $(\Omega ,B,\mu)$ se llama $espacio$ $de$ $medida$.
                   
                       
               \end{itemize} 
                    
     \section{DESAFIO } 
        
        
          \textbf{Teorema 1} $Los$ $siguientes$ $ afirmaciones$ $ son$ $ equivalentes$ $ para$ $ un$ $ grupo $ $G.$
                  \begin{multicols}{2}
                         \begin{enumerate}
                             \item $P(G)=1$ 
                              \item $G$ $es$ $abeliano$
                               \item $Z(G)=G$
                                \item $G'={1}$
                                 \item $C_G(a)=G$ $para$ $todo$ $a\in G$
                                  \item $G/G'\cong G.$                                  
                         \end{enumerate}
                   \end{multicols}  
         \textbf{Demostración.} Si $P(G)=1$, entonces $|L(G)|=|G|^2$. Luego $L(G)=G^2$, y esto significa $xy=yx$ para todo $x,y \in G$. Así $G$ es un grupoabeliano. Es inmediatoobservar que el razonamiento inverso tambien es cierto, lo que prueba que $1$, es equivalente a 2.
            \begin{flushright}
                  \textbf{\(\blacksquare\)}
            \end{flushright}
          \hspace{1 cm}Segun este resultado, para tener grados de conmutatividad diferentes de $1$ debemos analizar grupos no abelianos.
                
\end{document}
